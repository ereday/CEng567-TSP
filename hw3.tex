% Template LaTeX source file for homework problem solutions.
% Alan T. Sherman (9/9/98)

% Running LaTeX
%
% Name this file FOO.tex
% latex FOO
% latex FOO   
%    (You have to run latex twice to get the cross references correct.
%     Running latex creates a file FOO.dvi 
%     You can view dvi files with the program xdvi )
% xdvi FOO.dvi &
%
% lpr -d FOO.dvi
%    (To print the dvi file.   Be sure to use the "-d" print option,
%     and be sure your printer can handle dvi files (not all printers can).
%     Do NOT print with "lpr FOO.dvi", which will print tens of pages
%     of unreadable dvi source code. Printing a postscript (ps) file
%     is usually more reliable, as explained below.)
%
% dvips FOO.dvi
%    (To create a postscript file named FOO.ps 
%     which you can view with the program ghostview )
% ghostview FOO.ps &
% lpr FOO.ps
%    (To print the ps file.)

%%%%%%%%%%%%%%%%%%%%%%%%%%%%%%%%%%%%%%%%%%%%%%%%%%%%%%%%%%%%%%%%%%%%%%

\documentclass[12pt]{article}
\usepackage{amsmath,amssymb}
\usepackage{listings}

% Set the margins
%
\setlength{\textheight}{8.5in}
\setlength{\headheight}{.25in}
\setlength{\headsep}{.25in}
\setlength{\topmargin}{0in}
\setlength{\textwidth}{6.5in}
\setlength{\oddsidemargin}{0in}
\setlength{\evensidemargin}{0in}




%%%%%%%%%%%%%%%%%%%%%%%%%%%%%%%%%%%%%%%%%%%%%%%%%%%%%%%%%%%%%%%%%%%%%%%
% Macros

% Math Macros.  It would be better to use the AMS LaTeX package,
% including the Bbb fonts, but I'm showing how to get by with the most
% primitive version of LaTeX.  I follow the naming convention to begin
% user-defined macro and variable names with the prefix "my" to make it
% easier to distiguish user-defined macros from LaTeX commands.
%
\newcommand{\myN}{\hbox{N\hspace*{-.9em}I\hspace*{.4em}}}
\newcommand{\myZ}{\hbox{Z}^+}
\newcommand{\myR}{\hbox{R}}

\newcommand{\myfunction}[3]
{${#1} : {#2} \rightarrow {#3}$ }

\newcommand{\myzrfunction}[1]
{\myfunction{#1}{{\myZ}}{{\myR}}}


% Formating Macros
%

\newcommand{\myheader}[4]
{\vspace*{-0.5in}
\noindent
{#1} \hfill {#3}

\noindent
{#2} \hfill {#4}

\noindent
\rule[8pt]{\textwidth}{1pt}

\vspace{1ex} 
}  % end \myheader 

\newcommand{\myalgsheader}[0]
{\myheader{METU, Computer Engineering}
{CENG567 - Design and Analysis of Algorithms \\ Homework 3 - Pseudocodes \\ {\bf Erenay DAYANIK}  - 1819192 } {Fall 2015}{}}

% Running head (goes at top of each page, beginning with page 2.
% Must precede by \pagestyle{myheadings}.
\newcommand{\myrunninghead}[2]
{\markright{{\it {#1}, {#2}}}}


\usepackage{algorithm}
\usepackage{algpseudocode}

%%%%%% Begin document with header and title %%%%%%%%%%%%%%%%%%%%%%%%%
\makeatletter
\newenvironment{breakablealgorithm}
  {% \begin{breakablealgorithm}
   \begin{center}
     \refstepcounter{algorithm}% New algorithm
     \hrule height.8pt depth0pt \kern2pt% \@fs@pre for \@fs@ruled
     \renewcommand{\caption}[2][\relax]{% Make a new \caption
       {\raggedright\textbf{\ALG@name~\thealgorithm} ##2\par}%
       \ifx\relax##1\relax % #1 is \relax
         \addcontentsline{loa}{algorithm}{\protect\numberline{\thealgorithm}##2}%
       \else % #1 is not \relax
         \addcontentsline{loa}{algorithm}{\protect\numberline{\thealgorithm}##1}%
       \fi
       \kern2pt\hrule\kern2pt
     }
  }{% \end{breakablealgorithm}
     \kern2pt\hrule\relax% \@fs@post for \@fs@ruled
   \end{center}
  }
\makeatother
\begin{document}

\myalgsheader
\pagenumbering{gobble}
\pagestyle{plain}
\section{Backtracking}
\begin{breakablealgorithm}
  \caption{Algorithm for "Travelling Salesman Problem" using backtracking method [1]}
  \begin{algorithmic}[1]
    \State $//$ converts arg1 to reduced matrix
    \Procedure{REDUCE}{$arg1$}
  
    \State \textbf{Input:} $arg1$
    \State \textbf{Output:} $value$
    \\
    \State $m \gets arg1$.size();
    \State $value \gets  $ 0;
    \For{each $i\in [0,m-1]$}
      \State $min \gets   arg1$[$i,0];$
       \For{each $j\in [1,m-1]$}
	  \If{arg1$[$i,$j]>min$}
	  	 \State $min \gets arg1$[$i,j];$
	  \EndIf
	\EndFor	
	\For{each $j\in [0,m-1]$}
		\State arg1$[$i,$j] \gets  arg1$[$i,$j]- $min;$
	\EndFor
	\State $value \gets value$ + $min;$
    \EndFor
    \For{each $j\in [0,m-1]$}
	\State $min \gets arg1$[$ 0,j];$
	\For{each $i\in [1,m-1]$}
	 \If{arg1$[$i,$j]>min$}
	  	 \State $min \gets arg1$[$i,j];$
	  \EndIf	
	\EndFor
	\For{each $i\in [0,m-1]$}
		\State arg1$[$i,$j] \gets  arg1$[$i,$j]- $min;$
	\EndFor
	 \State $value \gets value$ + $min;$
    \EndFor
  \State \Return value

    \EndProcedure
    \State $//$ computes boundary function
    \Procedure{REDUCEBOUND}{$partialSolution$}
	 \State \textbf{Input:} $partialSolution$
	 \State \textbf{Output:} $ans$
		 \State \textbf{GlobalVariables:} $costMatrix$,$V$
	 \State $//$ $V$: all vertices in the graph
	 \State $//$ $M$: cost matrix stores weight of each edge in the graph
	 \State $//$ $partialSolution: [x_{0},x_{1},...,x_{m-1}]$
	 \If{$m=V$.size()}
		 \State \Return (COST($partialSolution);$
	 \EndIf
	\State $M^\prime[0][0] \gets  \infty;$
 	
	\For{each $y\in ${V$  \setminus   partialSolution$}}
		\State $M^\prime[0][j] \gets  {costMatrix[x_{m-1}][y]};$
		\State $j \gets j$ + $1;$
	\EndFor
	\State $i \gets 1;$
	\For{each $x\in ${V$  \setminus   partialSolution$}}
		\State $M^\prime[i][0] \gets  {costMatrix[x][x_{0}]};$
		\State $i \gets i$ + $1;$
	\EndFor
	\State $i \gets 1;$
	\For{each $x\in ${V$  \setminus   partialSolution$}}
		\State $j \gets 1;$
		\For{each $y\in ${V$  \setminus   partialSolution$}}
			\State $M^\prime[i][j] \gets  {costMatrix[x][y]};$
			\State $j \gets j$ + $1;$
		\EndFor
		\State $i \gets i$ + $1;$
	\EndFor
	\State $ans \gets REDUCE$($M^\prime);$

	
	    \For{$i=1$; $i<m$-1; $i++$ }
		      \State $ans \gets ans$ + costMatrix$[x_{i-1}][x_{i}];$
		  
	    \EndFor
		
	 \State \Return ans
  	 \EndProcedure	
	\\
	\\
	\\
	\\
	\\
	\\
	\Procedure{BACKTRACKING}{$level$,$partialSolution$}
	 \State \textbf{Input:} $level$,$partialSolution$
	\State $//$ $partialSolution$: $[x_{0},x_{1},...,x_{level-1}];$
	 \State \textbf{Output:} $OptX$,$OptC$
		 \State \textbf{GlobalVariables:} $C_{level}$ ($level$=0,1,2,...,n-1)
	 \State $//$ $C_{level}$: choice set for that level   
	\State $//$ $n$: number of nodes in the graph
	 \If{$level=n$}
		\State $C \gets  {COST(partialSolution)};$
		 \If{$C<OptC$}
			\State $OptC \gets  {C};$
			\State $OptX \gets  {partialSolution};$
		 \EndIf
	 \EndIf

	\If{$level=0$}
		\State$C_{level} \gets  {\{0\}};$
  	  	
	\ElsIf{$level=1$}
		\State$C_{level} \gets  {\{1,2,3,....,n-1\}};$
  	\Else  	 
		\State$C_{level} \gets  {C_{level-1} \setminus \{x_{level-1}\}};$
	\EndIf 
 
          \State$B \gets  {REDUCEBOUND(partialSolution)};$
          \For{each $x\in    C_{level}$}
		 \If{$B>=OptC$}
			\State \Return 
		 \EndIf
                   \State$x_{level} \gets  {x};$
		 \State $BACKTRACKING(level+1,partialSolution);$
	\EndFor
    \EndProcedure
  \end{algorithmic}
\end{breakablealgorithm}

\newpage
\section{Branch \& Bound}
\begin{breakablealgorithm}
  \caption{Algorithm for "Travelling Salesman Problem" using branch \& bound method [1]}
  \begin{algorithmic}[1]
   
    \Procedure{BRANCHandBOUND}{$level$,$partialSolution$}
	 \State \textbf{Input:} $level$,$partialSolution$
	\State $//$ $partialSolution$: $[x_{0},x_{1},...,x_{level-1}];$
	 \State \textbf{Output:} $OptX$,$OptC$
		 \State \textbf{GlobalVariables:} $C_{level}$ ($level$=0,1,2,...,n-1)
	 \State $//$ $C_{level}$: choice set for that level   
	\State $//$ $n$: number of nodes in the graph
	 \If{$level=n$}
		\State $C \gets  {COST(partialSolution)};$
		 \If{$C<OptC$}
			\State $OptC \gets  {C};$
			\State $OptX \gets  {partialSolution};$
		 \EndIf
	 \EndIf

	\If{$level=0$}
		\State$C_{level} \gets  {\{0\}};$
  	  	
	\ElsIf{$level=1$}
		\State$C_{level} \gets  {\{1,2,3,....,n-1\}};$
  	\Else  	 
		\State$C_{level} \gets  {C_{level-1} \setminus \{x_{level-1}\}};$
	\EndIf 
\State $count  \gets  {0};$
\For{each $x\in    C_{level}$}
                   \State$x_{level} \gets  {x};$
		 \State $nextchoice[count] \gets  {x};$
		\State $nextbound[count] \gets  {REDUCEBOUND(partialSolution)};$
		\State $count  \gets  {count+1};$
\EndFor

\State $//$ Sort nextchoice and nextbound so that nextbound is an increasing order 
\State $SORT(nextchoice,nextbound);$

   \For{$i=1$; $i<count$-1; $i++$ }
	\If{$nextbound[i]>=OptC$}
		\State \Return 
	  \EndIf 
		\State$x_{level} \gets  {nextchoice[i]};$	
		\State $BRANCHandBOUND(level+1,partialSolution);$
	    \EndFor
 \EndProcedure
  \end{algorithmic}
\end{breakablealgorithm}


\newpage
\begin {thebibliography}{9}

\bibitem{citation1}
  Donald L. Kreher , Douglas R. Stinson.
  \emph{Combinatorial Algorithms},
  CRC Press, New York,
  2nd edition,
  1999.

\end{thebibliography}
\end{document}



